\documentclass[12pt,reqno]{article}

\usepackage[usenames]{color}
\usepackage{amssymb}
\usepackage{amsmath}
\usepackage{amsthm}
\usepackage{amsfonts}
\usepackage{amscd}
\usepackage{graphicx}
\usepackage{mathrsfs}

\usepackage[colorlinks=true,
linkcolor=webgreen,
filecolor=webbrown,
citecolor=webgreen]{hyperref}

\definecolor{webgreen}{rgb}{0,.5,0}
\definecolor{webbrown}{rgb}{.6,0,0}

\usepackage{color}
\usepackage{fullpage}
\usepackage{float}

\usepackage{graphics}
\usepackage{latexsym}
\usepackage{epsf}
\usepackage{breakurl}

\setlength{\textwidth}{6.5in}
\setlength{\oddsidemargin}{.1in}
\setlength{\evensidemargin}{.1in}
\setlength{\topmargin}{-.1in}
\setlength{\textheight}{8.4in}

\newcommand{\seqnum}[1]{\href{https://oeis.org/#1}{\rm \underline{#1}}}

\newcommand{\N}{\mathbb{N}}
\newcommand{\Z}{\mathbb{Z}}
\newcommand{\R}{\mathbb{R}}
\newcommand{\lcm}{\mathrm{lcm}}
\let\up=\textsuperscript
\newcommand{\ct}{c}
\usepackage{bigints}
\def\restmod#1#2{#1\ (\mathrm{mod}\ #2)} %%% for the congruences


\begin{document}

\begin{center}
\epsfxsize=4in
\end{center}

\theoremstyle{plain}
\newtheorem{theorem}{Theorem}
\newtheorem{corollary}[theorem]{Corollary}
\newtheorem{lemma}[theorem]{Lemma}
\newtheorem{proposition}[theorem]{Proposition}

\theoremstyle{definition}
\newtheorem{definition}[theorem]{Definition}
\newtheorem{example}[theorem]{Example}
\newtheorem{conjecture}[theorem]{Conjecture}

\theoremstyle{remark}
\newtheorem{remark}[theorem]{Remark}

\begin{center}
\vskip 1cm{\Large\bf Remarks on a Nontrivial Permutation of the Natural Numbers
}
\vskip 1cm
\large
Robert Daland \\
Siri Natural Language Understanding \\
Apple, Inc \\
\href{mailto:r.daland@gmail.com}{\tt r.daland@gmail.com} 
\end{center}

\vskip .2 in

\begin{abstract}
some foo and some bar and some lorem ipso facto whatever 
\end{abstract}

\section{Introduction and notation}


     For blackboard bold symbols such as
$\mathbb Z$, $\mathbb Q$, $\mathbb R$, $\mathbb C$, use
{\tt \char'134mathbb\char'173Z\char'175}, for example.  You may need to include
the command {\tt \char'134usepackage\char'173amssymb\char'175}.

\medskip

\noindent Right:  {\tt And so the number of terms is \char'044n\char'044.} \\
\noindent Wrong:  {\tt And so the number of terms is \char'044n.\char'044} \\

\medskip

\subsection{Gcd and lcm}

     Be sure to use the built-in \TeX\ command {\tt\char'134gcd}
for greatest common divisor.  Don't write $(a,b)$ for the gcd of $a$ and
$b$; write $\gcd(a,b)$ instead.  For lcm, you will have to define
your own command so that it appears in the roman font.  The best way
to do this is to use the command

{\tt \char'134DeclareMathOperator\char'173\char'134lcm\char'175\char'173lcm\char'175}

Do {\it not\/} use square brackets for lcm!

\subsection{Binomial coefficients}

      Use {\tt\char'134choose} or {\tt\char'134binom}
for binomial coefficients.  
Do not use the latex array environment.

\subsection{Multi-letter functions}

     \textcolor{red}{As a general rule, all multi-letter functions such as $\sin, \cos, \tan$, etc., should appear in the roman font.}  For these functions you can use
the built-in \TeX\ commands {\tt\char'134sin, \char'134cos, \char'134tan}, etc.,
but for others (e.g., {\rm Li} for the logarithmic integral)
you may have to define your own commands.  Again, the best way to do this
is, e.g., 

{\tt \char'134DeclareMathOperator\char'173\char'134Li\char'175\char'173Li\char'175}

\subsection{Sequences}

Use parentheses, not braces, to denote sequences.  For example,
$(F_n)_{n \geq 0}$ is the correct way to write the Fibonacci sequence.


\section{Title page}

The title page should include the title of your article (capitalized),
and the complete postal mailing address and affiliations, including academic
department, e-mail address, postal code, and country,
for all authors.  



\section{Introduction}

Papers should have a numbered introductory section that provides motivation and
history of the problems discussed.  This is the place to put your results
in context, and summarize your main contributions.

\section{Abstract}

\end{document}

