\documentclass[12pt]{article}

\usepackage{fullpage}
\usepackage{amssymb}
\usepackage{amsmath}
\usepackage[usenames]{color}
\usepackage{graphicx}
\usepackage[colorlinks=true,
linkcolor=webgreen,
filecolor=webbrown,
citecolor=webgreen]{hyperref}

\definecolor{webgreen}{rgb}{0,.5,0}
\definecolor{webbrown}{rgb}{.6,0,0}
\definecolor{red}{rgb}{1,0,0}

\usepackage{color}


\begin{document}

\title{LaTeX Style Guide for the\\
{\it Journal of Integer Sequences}\\
\large Version 1.83}

\date{}
\author{}
\maketitle

	Authors of papers in the {\it Journal of Integer Sequences} should
write their papers in English.  
\textcolor{red}{If English is not your native language,
please ask a native-speaking colleague or local native speaker to
help proofread your paper.}    This will greatly improve your chances for
acceptance.  Probably you can find a native speaker at your university.
If not, you might try contacting your local US, Canadian, or British Embassy.

If you have no local access to a native speaker, you can (for a fee) have
people read and edit your paper online.  For example, visit \newline
\url{http://webshop.elsevier.com/languageservices/languageediting/} or
\newline
\url{http://www.edanzediting.com/}.

Again, if English is not your first language, you may want to consult
the book {\it Writing Mathematical Papers in English} by Jerzy Trzeciak,
which can be ordered (for 8 Euros) from \url{http://www.ems-ph.org} \ .  
Another good
resource is Trzeciak's Dictionary of Mathematical English, available
for free at \\
\centerline{\url{https://www.impan.pl/en/publishing-house/for-authors/dictionary}\ .}

If this is your first paper, or if you have simply not written many mathematical papers before, please read a guide on how to prepare mathematical papers
{\it before\/} sending us your submission.
One of the best is Steven G. Krantz, {\it A Primer of Mathematical
Writing}, 2nd edition, 2017.  It can be ordered from the
American Mathematical Society website.  If you do not wish to have the printed
copy, an electronic copy is available at
\url{https://arxiv.org/abs/1612.04888}.
Other good resources include
\begin{itemize}
\item a booklet written by Donald Knuth et al., and
available at \\ \url{http://jmlr.csail.mit.edu/reviewing-papers/knuth_mathematical_writing.pdf} 
\item an article of Halmos, available at \\
\url{http://www.domagoj-babic.com/uploads/ScholarlyStuff/Books/halmos.pdf} 
\item an article by Jerzy Trzeciak, available at \\
\url{https://www.impan.pl/wydawnictwa/dla-autorow/writing.pdf} \ .
\end{itemize}

Finally, we also recommend this (unfinished) advice of John Baez,
available here: \\  \url{https://math.ucr.edu/home/baez/boring.pdf} \, .

     Authors of papers in the {\it Journal of Integer
Sequences} should prepare their papers in LaTeX.  We do {\it not\/}
recommend use of
add-on packages such as Scientific Workplace.  We do not accept submissions
prepared with conversion tools, such as {\tt docx2latex}, as they do not
provide high-quality output.   Output prepared with packages like {\tt lyx}
is also likely to be of poor quality.
Please observe the
following guidelines.

      The guidelines most frequently violated appear in
\textcolor{red}{red}.  Please pay special attention to these.


\section{LaTeX advice}

     Please prepare your paper using the ``12pt'' option of latex, since this
is what we will use when we publish your paper.  Do not use ``11pt''
or ``10pt'', as this can cause problems formatting large equations when
we switch to ``12pt''.

     Use a single file for your latex source.  Do not spread your paper
over multiple latex files.

     Please do not submit papers that are based on the
proprietary style files
(e.g., {\tt mdpi.cls}) of other journals.  

    Please avoid the use of special-purpose packages or
macros whenever possible.
\textcolor{red}{Strip
your paper of references to all packages and definitions that you
do not actually use.}  (Do {\it not\/} just comment them out.)
\textcolor{red}{Remove all comments and commented-out lines (those with the symbol \%).}  There should be no occurrences of the symbol \% in your 
latex source file.

Do {\it not\/} use the {\tt MnSymbol} package, as it redefines basic
signs like equality, sum, inequality in unattractive ways.

{\bf It is probably worthwhile to download the latex file for a paper recently
published in the journal and model your paper on it.}  (However,
do {\it not\/} use the latex file for the instructions you are now
reading as a model!)  If you experience problems with the packages
we use (such as psfig), you can delete the corresponding line or just
comment it out.

    Do {\it not\/} include a date in the title page of your paper.

    Use the default (Computer Modern) font.  Do not use Times Roman or
other fonts.

    \textcolor{red}{Acknowledgments should be in a {\it separate, numbered
section} at the end of the paper.}

Avoid the use of PicTeX; it uses too many registers and is often not
compatible with packages we use to publish your paper.  If you
absolutely have to use it, consider the use of {\tt pictexwd} instead.

Avoid the use of pstricks; it often does not work correctly on our system.

     \textcolor{red}{Please do not use {\tt tcilatex} or the {\tt
amsart} article style when you prepare your article.}

Please examine the {\tt .log} file before you send anything to us.
This is where errors and warnings are listed.   Make sure you 
fix all significant errors, such as duplicate labels or right braces
without matching left braces, etc.   If you use TeXworks
you will need to remove the {\tt --clean} option in the preferences
in order to see the {\tt .log} file.

\section{Common grammatical errors}

\textcolor{red}{Please {\bf be sure to run your paper through a spell-checker before
submission!}}  Our Journal uses American spellings, so write
``recognize'' (not ``recognise'');
``generalize'' (not ``generalise''); ``color'' (not ``colour'');
``acknowledgment'' (not ``acknowledgement''), etc.

\begin{enumerate}

\item Avoid the passive voice.  Instead of saying ``In
[1] it is shown that all primes $> 2$ are odd'', say ``Smith [1] showed
that all primes $>2$ are odd''.

\item Avoid use of weak constructions such as ``this number'' or
``it''.  For example,

Wrong:  Let $x$ be a prime.  We now square this number.

Wrong:  Let $x$ be a prime.  We now square it.

Right:  Let $x$ be a prime.  We now square $x$.

\item Avoid the use of contractions, such as ``don't'', ``can't'',
``isn't'', etc.

Wrong:  The number $7$ is prime, since it isn't divisible by
$2, 3, 4, 5, $ or $6$.  

Right:  The number $7$ is prime, since it is not divisible by
$2, 3, 4, 5, $ or $6$.

\item The word ``precise'' is not a verb in English.

Wrong:  We now precise the connection between 
$\alpha$ and $\beta$.

Right:  We now make the connection between
$\alpha$ and $\beta$ more precise.

\item Use the word ``expansion'', not ``development''.

Wrong:  Let $[a_0, a_1, \ldots]$ be the continued fraction
development of $x$.

Right:  Let $[a_0, a_1, \ldots]$ be the continued fraction
expansion of $x$.

\item Use ``associate with'', not ``associate to''.

Wrong:  We now associate $x$ to $y$.

Right:  We now associate $x$ with $y$.

Wrong:  To each real number $x$ we associate a set $S_x$.

Right:  We associate a set $S_x$ with each real number $x$.

\item Use ``root'' for equations, and ``zero'' for polynomials. 

Wrong:  Let $\alpha$ be the positive root of $x^2 - x - 1$.

Right:  Let $\alpha$ be the positive zero of $x^2 - x - 1$.

Right:  Let $\alpha$ be the positive root of $x^2 -x - 1 = 0$.

\item Use the term ``pair'', not ``couple'', to denote two objects.

Wrong:  Let $(\alpha, \beta)$ be a couple of real numbers.

Right:  Let $(\alpha, \beta)$ be a pair of real numbers.

\item Use the construction ``We let $x$ denote $y$'' and not
``Denote by $x$ $y$''.

Wrong:  Denote by $\mathbb{N}$ the set of positive integers.

Right:  We let $\mathbb{N}$ denote the set of positive integers.

Right:  Let $\mathbb{N}$ denote the set of positive integers.

Right:  The set of positive integers is denoted by $\mathbb{N}$.

\item \textcolor{red}{Do not start sentences or clauses with notation.}

Wrong:  Since $x>0$, $x^2>0$ and the result follows.

Right:  Since $x>0$, we have $x^2>0$, and the result follows.

Wrong:  $p$ denotes a prime number.

Right:  The variable $p$ denotes a prime number.

It's fine to write things like ``The solutions are $a$, $b$, and $c$",
since here the commas do not introduce new clauses.

\item \textcolor{red}{Avoid run-on sentences.}
A run-on sentence (also called a ``comma splice'') is one that
expresses two thoughts in a single phrase, separated by
a comma.  Fix these by separating
into two or more sentences, or by connecting with a semi-colon
or a conjunction such as ``and''.   More information can be found
in the Wikipedia article on run-on sentences.  If you are not a native
speaker, or even if you are, you should learn to recognize and avoid
this common stylistic error.

Wrong:  Let $\Sigma$ be a finite alphabet, $\Sigma^*$ denote the
set of all finite words over $\Sigma$.

Right:  Let $\Sigma$ be a finite alphabet, and let
$\Sigma^*$  denote the set of all finite words over $\Sigma$.

Wrong:  Let $p$ be a prime number $\geq 3$, then $2^p \equiv 2$ (mod $p$).

Right:  Let $p$ be a prime number $\geq 3$.  Then $2^p \equiv 2$ (mod $p$).

\item Avoid treating citation numbers as objects of prepositions, or
subjects of sentences.
Treat them syntactically like footnotes.

Wrong:  [1] proves that $e$ is irrational.

Wrong:  In [1] it is proved that $e$ is irrational.

Wrong:  The article [1] proves that $e$ is irrational.

Right:  Euler [1] proved that $e$ is irrational.

\item Words like ``notation'', ``work'', ``progress'',
and ``information'' are mass
nouns in English, and as such, rarely appear in the plural.

Wrong:  We now introduce some definitions and notations.

Right:  We now introduce some definitions and notation.

\bigskip

Wrong:  You can find many works on continued fractions in the literature.

Right:  You can find many papers on continued fractions in the literature.

The Wikipedia article on mass nouns contains more information.

\item ``Any'' is a weak and sometimes ambiguous word in English.  For example,
if you say ``Condition (a) holds if $f(x) = 0$ for any $x$'', does it mean
that the condition holds if $f(x) = 0$ for {\it all\/} $x$, or 
{\it at least one\/} $x$?   Nearly always you can replace ``any'' with
either ``every'', ``all'', ``at least one'', or nothing at all, depending
on context.   Some examples follow:

Questionable:  Conjecture A holds for any prime $p$.

Can be replaced by:  Conjecture A holds for every prime $p$.

Questionable:   A solution exists for any $x$.

Can be replaced by:   A solution exists for all $x$.

Questionable:  For any prime number $p$ let $Z_p$ be the multiplicative group of invertible elements mod $p$.

Can be replaced by:  For prime numbers $p$ 
let $Z_p$ be the multiplicative group of invertible elements mod $p$.



\end{enumerate}

\section{Common punctuation errors}

\begin{itemize}

\item  Use colons properly.  
\textcolor{red}{In general, colons should not immediately follow
verbs.}

Wrong:  The resulting equation is:
$$ x = y^2 .$$

Right:  The resulting equation is
$$ x = y^2 .$$

Right:  The resulting equation is as follows:
$$ x = y^2.$$


\item  \textcolor{red}{Always put a comma after ``i.e.'' and ``e.g.''
and ``resp.''.}
Do {\it not\/} put these abbreviations in the italic font.

Wrong:  Let $x$ be a minimal element i.e. an element such
that if $y \leq x$ then $y = x$.

Wrong:  Let $x$ be a prime e.g. $2$. 

Right:  Let $x$ be a minimal element, i.e., an element such
that if $y \leq x$ then $y = x$.

Right:  Let $x$ be a prime, e.g., $2$.

\item Don't be stingy with commas.  Commas should set off 
parenthetical phrases such as ``for example'', ``in particular'',
and so forth.

Wrong:  Then $x$ for example is a real number.

Right:  Then $x$, for example, is a real number.

\item Avoid excessive and inappropriate capitalization.

Wrong:  We let $H(x)$ denote the Hankel Transform of $x$.

Right:  We let $H(x)$ denote the Hankel transform of $x$.

Wrong:  Tamigawa's Theorem states that $e^x = y$.

Right:  Tamigawa's theorem states that $e^x = y$.

Wrong:  Now we use the Cayley-Hamilton Theorem.

Right:  Now we use the Cayley-Hamilton theorem.

Wrong:  The result follows by the Prime Number Theorem.

Right:  The result follows by the prime number theorem.

Wrong:  The Fibonacci Numbers are numbers satisfying the
recurrence...

Right:  The Fibonacci numbers are numbers satisfying the
recurrence...

Wrong:  We use the Euclidean Algorithm to compute $\gcd(m,n)$.

Right:  We use the Euclidean algorithm to compute $\gcd(m,n)$.

Wrong:  This is an entry in Pascal's Triangle.

Right:  This is an entry in Pascal's triangle.

\item Lists of three or more things always need the ``Oxford comma''.

Wrong:  Smith, Jones and Wu solved the problem.

Right:  Smith, Jones, and Wu solved the problem.

\end{itemize}

\section{Common LaTeX errors}

     This section lists a few of the common errors made
when preparing papers in LaTeX.

\subsection{Blackboard bold}

     For blackboard bold symbols such as
$\mathbb Z$, $\mathbb Q$, $\mathbb R$, $\mathbb C$, use
{\tt \char'134mathbb\char'173Z\char'175}, for example.  You may need to include
the command {\tt \char'134usepackage\char'173amssymb\char'175}.

\subsection{Variables and expressions}

	\textcolor{red}{All mathematics must appear in mathematics mode.}

     Almost always, variables such as $x$, $y$, $n$, etc., should appear
in the italic font.  This will occur automatically if you remember
to enclose your equations (even references to a single variable) in
dollar signs or double-dollar signs, or use a latex equation environment.

\medskip

\noindent Wrong:  {\tt Let n be the number of integers in the list.}

\noindent Right:  {\tt Let \char'044 n\char'044\ be the number of integers in the list.}

\medskip

If a variable or expression ends a sentence or phrase written in 
a {\it non-displayed\/} environment,
do {\it not\/} include the
punctuation inside the {\tt \char'044 equation \char'044};
doing so messes up the spacing.    By contrast, in a displayed
environment, you must put the punctuation inside the environment.

\medskip

\noindent Right:  {\tt And so the number of terms is \char'044n\char'044.} \\
\noindent Wrong:  {\tt And so the number of terms is \char'044n.\char'044} \\

\medskip

Do not use mathematics mode for anything that is not a mathematical expression!
Do not use it for section numbers, numbers in itemized lists, etc.

\subsection{Spacing}

    Please try not to include commands that tweak the spacing (such as
{\tt \char'134\char'134},
{\tt \char'134noindent}, {\tt \char'134newpage},
{\tt \char'134bigskip}, {\tt \char'134pagebreak}, {\tt \char'134linebreak}, 
etc.) since when your paper is formatted for final publication, the page
breaks and spacing will probably be quite different from what you currently
see.    The proper way to separate paragraphs is with a single blank
line, and {\it not\/} with {\tt \char'134\char'134} at the end of the line.
However, do not put a blank line between lines of your main text and
any environment that follows, such as ``align''.

     \textcolor{red}{Don't forget that \textbf{if a period
follows a lower-case letter and is followed by a space, but does
not end a sentence}, then you must
put a {\tt \char'134} and then a space immediately 
after the period.}  For example:

\medskip

\noindent Wrong:
{\tt We use a flern (cf.\ the previous theorem) in the proof.} \\
\noindent Right:  
{\tt We use a flern (cf.\char'134\ the previous theorem) in the proof.} \\

\noindent Wrong:
{\tt We thank Dr.\ Smith for her assistance.} \\
\noindent Right:
{\tt We thank Dr.\char'134\ Smith for her assistance.}  \\

\noindent Wrong:
{\tt See Chan et al.\ \char'134cite\char'173smith\char'175\ for more information.}\\
\noindent Right:
{\tt See Chan et al.\char'134\ \char'134cite\char'173smith\char'175\ for more information.}\\

\noindent If you don't do this, there will be too much space
after the period, because LaTeX thinks it is the end of a sentence.

Again, just to be clear, it must be a period, then a backslash, then a space.
{\bf Do not put spaces between the period and the backslash!}

By the way,
there is no need to do this in the bibliography, because LaTeX automatically
turns off the ``extra space at the end of a sentence'' rule there.

\subsection{Accents}

    Be careful to use the proper accents.  The name
Erd\H{o}s, for example, uses a Hungarian accent, and
should be formatted with {\tt\char'134H}.  The name
Sierpi\'nski needs an accent on the ``n''.
Create accents using
the standard LaTeX abbreviations; do 
{\it not\/} use special non-ASCII characters, keyboard shortcuts,
or other exotic character sets to make them.  Warning: cutting and pasting
from web pages often results in non-ASCII 
characters being inserted into
your file, so avoid this practice.
The kinds of non-ASCII 
characters that often create problems
are alternate forms of quotation marks, long dashes (which look like ordinary
minus signs, but aren't), and accented letters.

{\bf Never use LaTeX math mode to create your accents.}

Here is how to do various accented letters: \\
	{\tt T\char'134'oth} gives T\'oth \\
	{\tt Ha{\char'134}v{\char'173}c{\char'175}ek} gives Ha\v{c}ek \\
	{\tt mis\char'134`ere} gives mis\`ere \\
	{\tt Schr\char'134"oder} gives Schr\"oder \\
	{\tt N\char'173\char'134o\char'175rg\char'173\char'134aa\char'175rd} gives N{\o}rg{\aa}rd


\subsection{Floor and ceiling}

    Be sure to use the built-in \TeX\ commands
{\tt \char'134lfloor}, {\tt \char'134rfloor} and
{\tt \char'134lceil}, {\tt \char'134rceil}, not square brackets,
when using these integer functions.

\subsection{Min and max}

     Be sure to use the built-in \TeX\ commands
{\tt \char'134min} and {\tt \char'134max} when using these functions.

\subsection{Gcd and lcm}

     Be sure to use the built-in \TeX\ command {\tt\char'134gcd}
for greatest common divisor.  Don't write $(a,b)$ for the gcd of $a$ and
$b$; write $\gcd(a,b)$ instead.  For lcm, you will have to define
your own command so that it appears in the roman font.  The best way
to do this is to use the command

{\tt \char'134DeclareMathOperator\char'173\char'134lcm\char'175\char'173lcm\char'175}

Do {\it not\/} use square brackets for lcm!

\subsection{Binomial coefficients}

      Use {\tt\char'134choose} or {\tt\char'134binom}
for binomial coefficients.  
Do not use the latex array environment.

\subsection{Multi-letter functions}

     \textcolor{red}{As a general rule, all multi-letter functions such as $\sin, \cos, \tan$, etc., should appear in the roman font.}  For these functions you can use
the built-in \TeX\ commands {\tt\char'134sin, \char'134cos, \char'134tan}, etc.,
but for others (e.g., {\rm Li} for the logarithmic integral)
you may have to define your own commands.  Again, the best way to do this
is, e.g., 

{\tt \char'134DeclareMathOperator\char'173\char'134Li\char'175\char'173Li\char'175}

\subsection{Parentheses}

      Use parentheses for grouping, not square brackets or braces.
You can get different sizes of parentheses using, for example,
{\tt \char'134bigl(} and {\tt \char'134bigr)}.  

     Do not use {\tt \char'134left(} and {\tt \char'134right)} unless
LaTeX gives you the wrong size parentheses (too small).

\subsection{Superscripts and subscripts}

{\it Never} put an object you are raising to a power inside braces (parens
are OK, of course).
To combine powers and indices, write {\tt x\_\char'173ij\char'175\char'136k} and
not {\tt \char'173x\_\char'173ij\char'175\char'175\char'136k}.
If you do the latter, then
the spacing is messed up, and it will appear like ${x_{ij}}^k$, which
is quite undesirable.

In general, try avoid subscripts of superscripts and superscripts of
subscripts; they make expressions hard to read.  Often this
can be done by adjusting the notation.  For example, instead of
$e^{x_n}$, you could write $\exp(x(n))$.   Instead of $a_{2^n}$, you could
write $a(2^n)$.

\subsection{Mod}

     Observe the distinction between the use of ``mod'' as a function of
two arguments, mapping $a \bmod b$ to the least non-negative residue
of $a$ modulo $b$, and ``mod'' as an equivalence relation.  For
the first, use the \TeX\ command {\tt\char'134bmod}.  For the second,
use the \TeX\ command {\tt\char'134pmod} for centered,
displayed equations \textit{only};
for in-line equations and subscripts write something like

\centerline{{\tt \char'044 x \char'134equiv a\char'044\ 
(mod \char'044b\char'044)},}

\noindent which typesets as follows: $x \equiv a$ (mod $b$).  Do not
use notation like $x \equiv y \ [p]$. 

      You can also define the following macro:

\centerline{\tt \char'134def\char'134modd\#1 \#2\char'173\#1\char'134~\char'134mbox\char'173\char'134rm (mod\char'175 \char'134~\#2\char'134mbox\char'173\char'134rm )\char'175\char'175}

\noindent which then can be used as follows:

\centerline{\tt \char'044 x \char'134equiv \char'134modd\char'173a\char'175~\char'173b\char'175\char'044 .}

The general rule to observe is that ``mod'' should {\it never\/} appear in the
italic font, even in theorem statements.

For chains of congruences, write
\def\modd#1 #2{#1\ \mbox{\rm (mod}\ #2\mbox{\rm )}}
$x \equiv y \equiv \modd{z} {w}$.

\subsection{Quote marks}

     Do not enclose words in ordinary quotation
marks {\tt "like this"}.   This results in the following
ugly output:
\begin{center}
"like this"
\end{center}
Instead, use the left-quote and right-quote
symbols, {\tt ``like this''}, which gives the correct
\begin{center}
``like this'' .
\end{center}

\subsection{Sequences}

Use parentheses, not braces, to denote sequences.  For example,
$(F_n)_{n \geq 0}$ is the correct way to write the Fibonacci sequence.

\subsection{Proper use of {\tt\char'134ldots} and {\tt\char'134cdots}}

     \textcolor{red}{Be sure to use {\tt\char'134ldots} and {\tt\char'134cdots} properly.}
The general rule is as follows:  you should use {\tt\char'134ldots} if
the center of mass of the items on either side is below the middle of
the line --- for example, if the items on either side are commas.  You
should use {\tt \char'134cdots} if the center of mass of the items on
either side is in the middle of the line --- for example, if the items
on either side are alphabet symbols.  In particular, {\tt \char'134cdots}
must be used for sums, products, and concatenations.

For example:

\medskip

Wrong:  Consider the product $a_1 a_2 \ldots a_n$.  \ \ \   (Here we used {\tt \char'134ldots}.)

Right:  Consider the product $a_1 a_2 \cdots a_n$.  \ \ \    (Here we used {\tt \char'134cdots}.)

\bigskip

Wrong:  Consider the sequence $a_1, a_2, \cdots, a_n$.  \ \ \   (Here we used {\tt \char'134cdots}.)

Right:  Consider the sequence $a_1, a_2, \ldots, a_n$.  \ \ \   (Here we used {\tt \char'134ldots}.)

\bigskip

\textcolor{red}{Under {\it no\/}
circumstances should you ever write ``...''.}  Use the appropriate
dots command instead.

\subsection{Proper punctuation of case statements}

\textcolor{red}{Do {\it not\/} use the {\tt array} environment to do case statements.}  
Please prepare case statements {\it exactly\/} as follows:
\medskip

\noindent{\tt
\char'134begin\char'173displaymath\char'175 \\
f(x) = \char'134begin\char'173cases\char'175 \\
\hphantom{XXX} 1, \char'046\ \char'134text\char'173if \char'044x\char'044\ is a rational number;\char'175 \char'134\char'134 \\
\hphantom{XXX} 2, \char'046\ \char'134text\char'173if \char'044x\char'044\ is a quadratic irrational;\char'175 \char'134\char'134 \\
\hphantom{XXX}  0, \char'046\ \char'134text\char'173otherwise.\char'175 \\
\char'134end\char'173cases\char'175 \\
\char'134end\char'173displaymath\char'175
}

\medskip

Note, in particular, the positions of the ampersand and semicolon,
and the use of {\tt \char'134text}.  This gives the following output.

\begin{displaymath}
f(x) = \begin{cases}
	1, & \text{if $x$ is a rational number;} \\
	2, & \text{if $x$ is a quadratic irrational;} \\ 
	0, & \text{otherwise.}
	\end{cases}
\end{displaymath}


\subsection{Words in set notation}

When using set notation, English words 
{\it must\/} appear in the Roman font.
The easiest way to use this is to use the {\tt \char'134text} command.

\subsection{Inequalities}

Please use {\tt \char'134geq}, not {\tt \char'134geqslant}.  Similarly,
please use {\tt \char'134leq}, not {\tt \char'134leqslant}.

\subsection{Emphasis}
Use italics for emphasis; for example, when introducing a new term.
In other circumstances bold might be appropriate.  However, do not use
underlining to emphasize text.

\section{Title page}

The title page should include the title of your article (capitalized),
and the complete postal mailing address and affiliations, including academic
department, e-mail address, postal code, and country,
for all authors.    
(By ``capitalized'' we do {\it not\/} mean you should capitalize every letter
of every word; just the first letter of all nontrivial words.)
Write your name with the surname {\it last}; if it is
unclear which is your first name and which is your surname, please
indicate this in a comment.

Do not include footnotes to the title.  Sponsoring information can be
placed in a footnote attached to the individual author's name.

Lines of your address should not end in commas.

\section{Sections}

Break your paper up into logical sections.  
\textcolor{red}{Section titles should
be capitalized like an ordinary English sentence; do not add extra
capitalization.}

\section{Footnotes}

We strongly
discourage the use of footnotes.  Incorporate the text of footnotes, to the
extent possible, in the main text.

\section{Definitions}

Terms that are being defined should be in a special font, such as italic
or slant. 

For example,

\centerline{A {\it flern} is a 3-dimensional hypersquare.}

Avoid introducing new terms and notation when there are already accepted
equivalents widely in use in the mathematical community.
For example, for the Fibonacci numbers, you
should use the notation $F_n$, and the numbers defined by
$F_0 = 0$, $F_1 = 1$, $F_n = F_{n-1} + F_{n-2}$ for $n \geq 2$.

\section{People}

\textcolor{red}{When referring to people, \textbf{use their last name only}, unless additional
information is required to disambiguate.}  If you {\it do\/} include initials,
make sure there is a space between each initial and between the initials
and the name.  

\medskip

\noindent Right:  Euler proved that $e$ is irrational. \\
\noindent Wrong:  L. Euler proved that $e$ is irrational.

\medskip

\noindent Right:  J. R. Smith \\
\noindent Wrong:  J.R. Smith \\
\noindent Wrong:  J R Smith \\
\noindent Wrong:  John R Smith

\section{Theorems}

\textcolor{red}{Use the {\tt \char'134begin\char'173theorem\char'175} $\ldots$
and {\tt \char'134end\char'173theorem\char'175} environments for theorems,
lemmas, propositions, remarks,
etc.}  Theorems should be numbered.  Refer to theorems,
lemmas, propositions, remarks, sections, equations, figures, tables, etc.\ using
labels; \textcolor{red}{do {\it not} hard-code references to them.}  When you
refer to theorems, definitions, propositions, and so forth, be sure
to capitalize the word Theorem (resp., Definition, Proposition, etc.) if
it is attached to a reference label (number), and not otherwise.

Do not put space characters or special characters, such as minus signs,
in labels!

\noindent  Right:  We now use Theorem 4.\\
\noindent Wrong:  We now use theorem 4.\\
\noindent Right:  We now use a previous theorem.\\
\noindent Wrong:  We now use a previous Theorem.\\

To get proper definitions, use the
{\tt \char'134usepackage\char'173amsthm\char'175} command.

Do not redefine equation numbers or appearance.  

Here is the code we use for declaring theorem environments:

{\tt
\begin{verbatim}
\theoremstyle{plain}
\newtheorem{theorem}{Theorem}
\newtheorem{corollary}[theorem]{Corollary}
\newtheorem{lemma}[theorem]{Lemma}
\newtheorem{proposition}[theorem]{Proposition}

\theoremstyle{definition}
\newtheorem{definition}[theorem]{Definition}
\newtheorem{example}[theorem]{Example}
\newtheorem{conjecture}[theorem]{Conjecture}

\theoremstyle{remark}
\newtheorem{remark}[theorem]{Remark}
\end{verbatim}
}

You can just cut and paste this into your file, right after the begin
document command.

\section{Equations}

Not all equations need to be numbered.  If you number an equation,
use a label and then refer to the label using
{\tt \char'134eqref\char'173eq1\char'175} or
{\tt Eq.\char'176(\char'134ref\char'173eq1\char'175)}
something similar.  Do not use things like (*), with a
star or asterisk, to number equations.

For multiple related equations on consecutive lines,
please use the {\tt align} environment.  \textcolor{red}{When you do so, remember
that the {\tt \&} symbol should {\it precede\/} the relational symbol
({\tt =} or {\tt >} or $\ldots$).}
Do not use {\tt eqnarray}, as it produces bad spacing.

\section{Definitions, examples, and remarks}

All definitions, examples, and remarks should be stated in the roman font,
except (of course) for 
mathematical symbols.  You can use the following code as an example.

\noindent{\tt \char'134theoremstyle\char'173definition\char'175} \\
\noindent{\tt \char'134newtheorem\char'173defn\char'175\char'173Definition\char'175} \\

\section{Proofs}

\textcolor{red}{Use the commands {\tt \char'134begin\char'173proof\char'175} and
{\tt \char'134end\char'173proof\char'175} to delimit proofs.}
These are available in the {\tt amsthm} package mentioned above.
Do not change the appearance of the proof environment.

\section{Tables and figures}

Tables and figures should be {\bf centered} on the page, using the {\tt
center} environment.  Each table and each figure should have a number.
Captions should appear {\it underneath} the table or figure, and
end in a period.  Use {\tt
rescalebox}, if necessary, to make sure your table fits properly on the
page.

Note that the position of tables and figures could change as your paper
is reformatted for final publication.  If the positioning is crucial, then
please use the {\tt float} latex package in your preamble, and use the
{\tt H} option to force the table/figure to appear in the place you
need it.    

\section{Introduction}

Papers should have a numbered introductory section that provides motivation and
history of the problems discussed.  This is the place to put your results
in context, and summarize your main contributions.

\section{Abstract}

Every paper should have a short abstract of 50 to 200 words, written
in the present tense.  The purpose of an abstract is to summarize what
you did {\it in general terms only}.  
\textcolor{red}{The abstract
should be free of
symbols and equations to the extent it is possible.}

When referring to results you prove in the paper, use the present tense.  
\textbf{Avoid the passive voice in abstracts, wherever reasonable.}

The abstract should be an independent entity and should stand on its
own.  For example, it {\it should not\/} contain citations to the
bibliography, or references to the numbers of equations, theorems, or
sections of the paper.  It should not contain numbered equations
itself.   

When referring to other work in the abstract, you can refer to author's
last names, but avoid mentioning years, journal names, or other
information.

Similarly, the paper text itself should not rely in any way on definitions
or notation introduced only in the abstract.

\section{Sequence numbers}

Be sure to include sequence numbers from Sloane's {\it On-Line Encyclopedia
of Integer Sequences} for all sequences you discuss in your paper.
The list of all such sequences should be summarized at the end of your
paper, sorted in ascending order.    If the sequences do not exist in the
{\it Encyclopedia}, please submit them to {\tt www.oeis.org}
and record the A-numbers assigned,
and add those to your paper.

When you refer to a sequence inside your paper, use the ``seqnum'' macro:
\begin{verbatim}
\newcommand{\seqnum}[1]{\href{https://oeis.org/#1}{\rm \underline{#1}}}
\end{verbatim}


\section{Citations}

     \textcolor{red}{Use citations syntactically like footnotes,
not as objects of prepositions.}
Avoid saying things like ``In [1] we find the following result.''  Instead,
say ``Jones [1] proved the following result.''  Use the LaTeX command
{\tt \char'134cite}; do {\it not\/} hard-code references to the bibliography.

	Avoiding enclosing citations in an extra pair of parentheses. 
In other words, citations should appear like ``[13]'' and not ``([13])'' or
``(see [13])''.

If you cite a paper with many authors inside the
text, you can use ``et al.'', but do
not put it in italics and use the first author's name.  ({\bf However, be sure
to give the complete author list in the bibliography.})

      In the bibliography, if the author has two initials, be sure
to place a space between the two initials. 

Wrong:  N.J.A. Sloane

Right:  N. J. A. Sloane

\medskip

      Two authors should be separated with ``and'':

Wrong:  J. Smith, D. Jones

Right:  J. Smith and D. Jones

\medskip

      Three or more authors should be separated with the ``Oxford
comma''.

Right:  J. Smith, D. Jones, and Z. Xu

Wrong:  J. Smith, D. Jones and Z. Xu

\medskip

      When simultaneously citing multiple references, use syntax
similar to {\tt \char'134cite\char'173ref1,ref2,ref3\char'175} to combine
all references in a single pair of brackets; do {\it not\/} write
{\tt \char'134cite\char'173ref1\char'175, \char'134cite\char'173ref2\char'175,
\char'134cite\char'173ref3\char'175}.

      When citing a theorem or page number in another work, say
{\tt \char'134cite[p.\char'134\ 123]\char'173ref1\char'175} or something
similar.  Note in particular the backslash and space after the dot.  This is
needed because LaTeX assumes that a dot following a lowercase letter indicates
the end of a sentence, and hence inserts extra space.

     Please use the following examples when preparing citations.
Pay careful attention to punctuation and the use of roman, italic,
and bold fonts.   In particular, notice that page ranges should be separated
by two hyphens in LaTeX:  write {\tt 123--145}, not {\tt 123-145}.

Please use the standard {\it Mathematical Reviews}
abbreviations for journal names, with the exception that for particularly
obscure journals you may provide the entire name.

     The {\it Mathematical Reviews} journal abbreviation list can be
found here:

\centerline{\url{http://www.ams.org/msnhtml/serials.pdf}}

     Do not include citations to reviews of the articles, such as those
appearing in {\it Zentralblatt} or {\it Math.\ Reviews}.

     Avoid references to secondary sources, such as Wikipedia, unless
there is really no alternative.

Always give a complete author list in the bibliography.  Be sure that
all mathematics in bibliography items appears in mathematics mode, like
in the main text.

\subsection{Article citation}

    1.  J. Chan and F. E. Smith, An article about Chan-Smith numbers,
    {\it J. of Chan-Smith Numbers} {\bf 13} (1998), 123--124.

\smallskip

Provide the volume, but {\bf not} the issue number, unless the issue
number is required to uniquely specify the paper.  Note that words in article
titles should {\it not\/} be capitalized, with the following exceptions:
the first word, proper nouns, and German nouns.    The journal name should
be in italics; the volume number should be in bold.  Do not use ``pp.'' to
provide page numbers for articles.   Use {\tt \char'055\char'055}
for page ranges.

Please use the standard {\it Mathematical Reviews}
abbreviations for journal names, with the exception that for particularly
obscure journals you may provide the entire name.

\subsection{Book citation}

    2.  A. Alces, {\it Introduction to Moose Theory}, Springer, 1995.

\smallskip

Book titles should be in italics.
Note that words in book titles should be capitalized, with the exception of
very short unimportant words, such as ``to'', ``of'', ``and'', etc.
Do not include the ISBN number.  It is not necessary to give the place
of publication unless it is a very rare or hard-to-find book.

If you cite a particular theorem or page or section inside a book,
then use the bibliography to list the book information {\it only}.
When you cite it, however, you should use syntax like
{\tt \char'134cite[Thm.{\char'134} 2.3,
p.{\char'134} 45]\char'173Alces\char'175}
to get something like [17, Thm.\ 2.3, p.\ 45].  Avoid citing a book
without specifying the exact result you are using.

\subsection{Article in conference proceedings or book}

    3.  B. Franklin, The public library as an aid to research,
    in G. Washington and T. Jefferson, eds., {\it Public Libraries in
    the United States}, Addison-Wesley, 2001, pp.\ 16--32.

\medskip

\noindent    4.  P. Flajolet, How to count, in {\it Automata, Languages, and
    Programming:  Proc.\ ICALP 1990}, Lect.\ Notes in Comp.\ Sci.,
    Vol.\ 443, Springer, 1991, pp.\ 220--234.

\smallskip

Capitalize the name of the book,
but {\it not\/} the paper you are referring to
in the book or the series.
Note that here, unlike the case of a journal article, the
abbreviation ``pp.'' is used.

\subsection{Unpublished material or material on the web}

     5.  B. Obama, G. Bush, and W. J. Clinton,
     Combinatorial reasoning in American elections,
	 preprint, 2005,  \url{http://www.barackobama.com/combin.pdf}.

\medskip

\noindent     6.  J. Schmoe, Pattern avoidance, arxiv preprint arXiv:1111.2222 [math.NT], 2010. Available at \url{http://arxiv.org/abs/1111.2222}.

\smallskip

You should use the command {\tt \char'134url} to specify the URL of
electronic manuscripts.  (This command is available in
the {\tt hyperref} package.)  When referring to an arxiv paper please
cite the landing page (abs) of the paper, not the pdf directly.

Note that the correct URL for the Online
Encyclopedia of Integer Sequences is \url{https://oeis.org}.

\section{Other issues}

All sections of your paper should be numbered.  Do {\it not\/}
hard-code references to section numbers; give each section a label
and refer to it.

Please be sure that your paper contains a list of {\it key words and
phrases} and the appropriate {\it AMS 2020 Mathematics Subject
Classifications}.  The key words should be in the singular (e.g., write
``Fibonacci number'' and not ``Fibonacci numbers''), should be separated
by commas, and should not be capitalized.
A list of all the subject
classifications can be found at\newline
\centerline{\url{https://mathscinet.ams.org/msnhtml/msc2020.pdf} \ .}

\noindent {\bf Provide only
one classification as primary} and any additional ones as secondary.

Avoid starting a line of your file  with the word ``From''.  Many mailers
insert a $>$ character in such lines, causing a question mark to appear
in your text.  If you must start a line of the file with the
word ``From'', you can insert a space first.

\textcolor{red}{Do not include non-ASCII special characters in your file.}  These can arise,
for example, from cutting and pasting references from the web.  They
create incorrect output.
The most common mistakes are the following:  wrong quotes or apostrophes,
dashes not represented by
{\tt --} or {\tt ---} but by other characters, and accented special
characters.

To learn how to detect non-ASCII characters, read \\
\url{https://tinyurl.com/2t8jtydm}.  Or you can simply paste the source
of your paper in this web page:  \url{https://pages.cs.wisc.edu/~markm/ascii.html}.



\end{document}
